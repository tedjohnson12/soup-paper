% Define document class
\documentclass[twocolumn]{aastex701}
\usepackage{showyourwork}
\usepackage{mhchem}

% Begin!
\begin{document}

\newcommand{\refp}{(ref?)}
\newcommand{\reft}{ref?}
\newcommand{\oz}{O$_3$}
\newcommand{\todo}{\textcolor{orange}{{\bf TODO}}}

\newcommand{\diam}{\variable{output/inst-diam.txt}}
\newcommand{\lifetime}{\variable{output/inst-lifetime.txt}}
\newcommand{\ucen}{\variable{output/inst-center_u.txt}}
\newcommand{\bcen}{\variable{output/inst-center_b.txt}}
\newcommand{\vcen}{\variable{output/inst-center_v.txt}}

\newcommand{\uwidth}{\variable{output/inst-width_u.txt}}
\newcommand{\bwidth}{\variable{output/inst-width_b.txt}}
\newcommand{\vwidth}{\variable{output/inst-width_v.txt}}

\newcommand{\uthrough}{\variable{output/inst-through_u.txt}}
\newcommand{\bthrough}{\variable{output/inst-through_b.txt}}
\newcommand{\vthrough}{\variable{output/inst-through_v.txt}}

\newcommand{\salgamma}{\variable{output/targets-gamma.txt}}
\newcommand{\fov}{\variable{output/inst-fov.txt}}


% Title
\title{The Stratospheric Ozone Ultraviolet Photometer (SOUP) Mission}

% Author list
\author{Ted Johnson}
\affiliation{Nevada Center for Astrophysics}
\affiliation{Department of Physics and Astronomy, University of Nevada, Las Vegas}
\email{ted.johnson@unlv.edu}

% Abstract with filler text
\begin{abstract}
    We present the Stratospheric Ozone Ultraviolet Photometer (SOUP) mission concept, a space telescope designed to detect the biosignature ozone in the upper atmospheres of rocky exoplanets. SOUP combines the sensitivity of transmission spectroscopy with the efficiency of a photometric survey by taking high-precision, full field-of-view images in three photometric bands.
\end{abstract}

\section{Introduction}
One of the most powerful methods for characterizing the composition of exoplanet atmospheres to date is transmission spectroscopy \citep[e.g.][]{garcia2022,may2023,moran2023,cadieux2024,fournier-tondreau2024}. Despite this technique's unique ability to probe the gasses in the upper atmospheres of a wide variety of exoplanets, its immense cost prevents its meaningful use on rocky, habitable-zone (HZ) exoplanets which would require hundreds of observations to reach the signal-to-noise ratio (SNR) required to detect a biosignature.

Conversely, thousands\footnote{\url{https://exoplanetarchive.ipac.caltech.edu/docs/counts_detail.html}} of exoplanet have been found via the transit method because it relies on photometry -- a dedicated mission (e.g. Kepler \citep{borucki2004} or TESS \citep{winn2024}) can observe thousands of targets at once. I propose a new mission to combine the best elements of each technique in order to efficiently explore a whole population of Earth-like planets in search of the biosignature ozone. I call this mission the Stratospheric Ozone Ultraviolet Photometer, or SOUP. Figure \ref{fig:soup-spec} demonstrates the basic idea behind detecting \oz{} through photometry: if we measure the transit depth simultaneously in multiple bands, we can infer the presence of absorbing gasses in the atmosphere. When we chose these bands carefully, we can become sensitive to potential biosignatures such as \oz{}.

\begin{figure}
    \includegraphics[width=0.5\textwidth]{figures/spectrum.pdf}
    \caption{Transmission spectra and band-integrated transit depths for two sets of potential SOUP targets. {\bf Top}: Earth-sized planet orbiting an M-dwarf at a distance of 10 parsecs from the solar system. The green ``Has \oz'' model has an ozone volume-mixing ratio (VMR) of $10^{-4}$. Both models include a background \ce{N2} atmosphere. {\bf Bottom}: Sub-Neptune orbiting a Sun-like star at a distance of 200 parsecs. Like the panel above, ``Has \oz'' means a VMR of $10^{-4}$, but the background gas in this case is \ce{H2}. In both panels, the planet has an isothermal 300 K atmosphere and an opaque surface at 1 bar. The noise in these models is calculated based on the number of transits observed over the \lifetime{}-year mission lifetime. The slope that appears in all spectra is due to scattering of the background gas and behaves predictably. Spectra produced by the Planetary Spectrum Generator \citep{villanueva2018}.}
    \label{fig:soup-spec}
    \script{spectrum.py}
\end{figure}

\subsection{Ozone as a biosignature}
Ozone on Earth is primarily produced as a photochemical byproduct of molecular oxygen (\ce{O2}), and as an indicator of biotically produced oxygen it has been discussed as a molecule of significant interest for the future Habitable Worlds Observatory \citep[see][]{latouf2024a}. Specifically, ozone produces strong absorption even at trace abundance, and is believed to be {\em more} detectable on planets with high altitude clouds that otherwise obscure molecular signatures from reaching astronomer's telescopes \citep{kofman2024}. 

\subsection{Abiotic Ozone}
Despite being an indicator of biotic oxygen, ozone can also be produced when geologic processes produce \ce{O2} in small amounts. \citet{domagal-goldman2014} found that detectable levels of \ce{O3} and \ce{CH4} can coexist in some abiotic scenarios. Those authors suggest that an ozone detection is not necessarily a biosignature, but that follow-up observations at infrared wavelengths could resolve that degeneracy.

\subsection{Transit Spectroscopy in the Era of JWST \& HST}
Review of James Webb Space Telescope (JWST) rocky planet transit spectroscopy by \citet{kreidberg2025} found that JWST had successfully detected signals on the order of $\sim3$ scale heights ($H$). Ozone does have an absorption feature at $9.5\,\mathrm{\mu m}$, well within the MIRI Low Resolution Spectroscopy (LRS) bandpass, that is strong enough to produce the $5 H$ signal that \citeauthor{kreidberg2025} argue should be the standard for atmosphere detection. However, observation time limits (to, at maximum, several dozen transits) mean that there are very few targets where a low-enough signal-to-noise ratio (SNR) could be achieved to make this detection feasible.

The Hubble Space Telescope (HST) similarly has been used to observe planetary transits, but at UV, optical, and near-infrared wavelengths \citep[for examples, see][, respectively]{behr2025,lothringer2018,fu2017}. However, it mainly looked at giant planets because of limited observing time. 

\subsection{Multiband Photometry}
Multi-wavelength photometry is not new to instrument design. For example, the ESA's Euclid mission \citep{euclidcollaboration2025} will include four photometric bands between 0.5 and $2\,\mathrm{\mu m}$. The GAIA mission \citet{gaiacollaboration2016} has performed a three-band photometric survey of the whole sky out to 20$^\text{th}$ magnitude in its G band. JWST's NIRCam instrument \citep{rieke2005} can take simultaneous short ($0.6-2.3\,\mathrm{\mu m}$) and long ($2.4-5.0\,\mathrm{\mu m}$) images because of a dichroic mirror in its optical path.

\section{Mission Design}

\subsection{Science Goals}

\subsubsection{Primary Goal: Detection of Ozone}
The SOUP mission detects ozone by targeting three specifically-chosen photometric bands for transit-depth observation. Two of these bands, U and V, are at wavelengths where ozone opacity is very high. The third band, V, is in a region of the spectrum with no ozone opacity and acts as a measure of the continuum with which the ozone opacity depths are compared.

Ozone opacity is strongest in the U band, but the majority of nearby stars are low-mass and cool, and so they do not emit very much UV light. Therefore, the V band is necessary to characterize ozone in M-dwarf systems. These three bands are chosen to maximize the ozone sensitivity while minimizing sensitivity to potential background gases. Figure \ref{fig:gas} shows that trace amounts of ozone dominate the transit signal for a variety of gases. This is expected -- optical wavelengths fall in an energy regime with few molecular transitions.

\begin{figure}
    \includegraphics[width=0.5\textwidth]{figures/gases.pdf}
    \caption{Trace ozone signal for a variety of background gasses. At optical and near-UV wavelengths, the dominant opacities are just ozone and Rayleigh scattering, with only minor contributions from background gases. In all models the \oz{} VMR is $10^{-4}$} and the planet is a HZ M-Earth at 10 pc.
\end{figure}

\subsection{Secondary Goal: Planet Discovery}

A byproduct of searching transiting planets for ozone is that many transiting planets will be discovered that are not our primary targets. The Kepler mission discovered over 4000 planets, the vast majority of which have orbital periods less than 100 days \citep{lissauer2023}. SOUP's field-of-view is four times Kepler's, so we can expect to find $\sim 16\,000$ planets in the first 4 years of observations. The mission, however will last for \lifetime years -- enough time to increase sensitivity to longer-period transiting planets in that field. \citet{lissauer2023} was able to calculate statistically significant planet frequencies for Earth-sized planets out to period of 128 days. With a 2.5 times longer baseline, we can expect to do similar statistics on small rocky planets out to periods of about 1 year. Finding more long period (and high angular separation) planets provides an opportunity to investigate potential targets for the upcoming Habitable Worlds Observatory (WHO), which will focus mainly on Earth-sized planets around Sun-like stars within 30 pc \citep[see][]{tuchow2024a,tuchow2025}.

% \subsection{Secondary Goal: Stellar Characterization}
% \todo

% \subsection{Secondary Goal: Transient Event Science}
% \todo

\subsection{Instrument}

The SOUP spacecraft will consist of a \diam{} aperture mirror that feeds light through a series of dichromatic beam splitters. The first splitter reflects UV light through the U filter onto a detector. The transmitted light passes through a second splitter, which transmits light short of 500 nm. The reflected light passes through the V filter and transmitted light through the B filter, both beams landing on an optical detector. These three photometric bands are described in Table \ref{tab:bands}
\begin{table}
    \centering
    \begin{tabular}{c|ccc}
        \hline
        Band & Center & Width & Throughput \\
        \hline
        U & \ucen & \uwidth & \uthrough \\
        B & \bcen & \bwidth & \bthrough \\
        V & \vcen & \vwidth & \vthrough \\ \hline \hline
        
    \end{tabular}
    \caption{The SOUP photometric bands. The throughput column accounts of the efficiency of the optics and detector quantum efficiency averaged over the bandpass.}
    \label{tab:bands}
\end{table}

The SOUP field of view is approximately four times that of Kepler, so to achieve similar angular resolution, the detector array requires $384\; 1024~\times~2200$ pixel CCD detectors -- 128 for each band. These are arranged so that the arrays are approximately square.

\subsection{Orbit \& Logistics}

Similar to the Kepler spacecraft \citep{borucki2004}, the SOUP observatory will be launched into a heliocentric orbit that trails Earth. A trailing orbit is preferable to low-Earth-orbit because it allows the FOV to be observed continuously and it removes the potential for day/night airglow cycles to contaminate the timeseries data.

\subsection{Targets}
Before choosing a specific field for the SOUP mission, I consider an idealized ``typical'' field. Assume the density of stars ($\rho$) is constant and approximately $0.05 \; \mathrm{{M_{\odot}}~{pc^{-3}}}$. For a field-of-view (FOV) $\Omega$ and distance $d$, the enclosed stellar mass is
\begin{align}
    \label{eq:mtot}
    M_\text{tot} = \Omega d^3 \rho \, .
\end{align}

The Salpeter mass function tells us that the number of stars with masses between $m$ and $m+dm$ is \citep{salpeter1955}
\begin{align}
    N\,dm = A~m^{-\alpha}~dm\, ,
\end{align}
where $m\equiv M/\mathrm{M_\odot}$ and $\alpha$ was measured to be 2.35. We normalize this function by assuming the total mass from equation (\ref{eq:mtot}) and considering stars from $0.1-2\; \mathrm{M_\odot}$. We find
\begin{align}
    A = \salgamma~\Omega ~d^3~\rho~\mathrm{M_\odot}^{-1}\, ,
\end{align}
and
\begin{align}
    N\,dm = \salgamma~\Omega ~d^3 \rho~\mathrm{M_\odot}^{-1} ~m^{-\alpha}\, .
\end{align}

The number of stars within a distance $d$ that have masses greater than $m_0$ is then
\begin{align}
    N(m>m_0) &= \salgamma~\Omega~d^3~\rho~\int_{m_0}^2 m^{-\alpha} dm \\
    &= \salgamma~\Omega~d^3~\rho~ \frac{2^{1-\alpha} - m_0^{1-\alpha}}{1-\alpha}\, . \\
\end{align}

\begin{figure}
    \includegraphics[width=0.5\textwidth]{figures/distance.pdf}
    \caption{Detectability of stratospheric ozone as a function of distance and host star spectral type. $\chi^2_\text{red}$ measures the mismatch between simulated measurements with and without ozone, with $\chi^2_\text{red} > 3$ signifying a statistically significant difference between the two models. Stellar spectral types are varied from very cool M8 (similar to TRAPPIST-1) to hot and massive A0. The G2 star is solar-like. The two planetary models considered are HZ Super-Earth-like ($1.5\;\mathrm{R_\oplus}$), with an \ce{N2} atmosphere, and a HZ Neptune, with an \ce{H2} atmosphere. The small signals produced by the Earth-sized planet make ozone detectable only out to 20 pc for M5 stars, whereas habitable Neptunes may host detectable ozone out to 1000 pc for all stellar types.}
    \label{fig:distance}
    \script{distance.py}
\end{figure}

The ratio between signal (determined by planet size) and noise (determined by the number of stellar photons received) determines detectability, and so, to determine the number of suitable targets we can ask: For planet scenario X, at what distance does ozone become undetectable around a star of mass $M$? And how many of those stars are there? Figure \ref{fig:distance} shows that We are sensitive to super-Earths out to 20 pc and Neptunes to 1000 pc, regardless of stellar type. There are about 0.3 targets $\mathrm{deg}^{-1}$ within 20 pc and 34,000 targets $\mathrm{deg}^{-1}$ within 1000 pc. It is safe to assume that the fraction of HZ planets which transit their star is on the order of 1\%, meaning that with a TESS-like FOV of \fov{}, we can expect to constrain ozone on $\sim 6$ super-Earths within 20 pc. At the same time, that FOV contains over 7 million stars within 1000 pc for characterization of HZ Neptune-sized planets. \citet{lissauer2023} used Kepler data to find that about 10\% of stars host a $2-4\;\mathrm{R_\oplus}$ planet with instellation similar to Earth, and so we can expect to find $\sim 7000$ transiting potentially-habitable Neptunes and sub-Neptunes in our FOV.




\subsection{Data Processing}

Each full CCD image contains 2.1 GB of data -- an amount unfeasible to downlink from a spacecraft in a high cadence. Similar to Kepler, SOUP will process the raw data on board the spacecraft and downlink lightcurves for pre-selected targets. With 7 million potential targets and a 15-minute cadence, the bandwidth required to downlink the processed lightcurves is 62 kBi/s.

Once on the ground, the lightcurves will be corrected for instrumental and astrophysical variability using the same differential photometry system that has been so successful for Kepler. Standard stars will be selected as targets and the data will be continuously calibrated using these standards. Importantly, each band will be corrected independently to avoid bias.

\section{Budget}
\todo

\section{Timeline}
\todo


\bibliographystyle{aasjournal}
\bibliography{vspec,soup}

\end{document}

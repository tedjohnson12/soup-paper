% Define document class
\documentclass[twocolumn]{aastex701}
\usepackage{showyourwork}
\usepackage{mhchem}

% Begin!
\begin{document}

\newcommand{\refp}{(ref?)}
\newcommand{\reft}{ref?}
\newcommand{\oz}{O$_3$}

\newcommand{\diam}{\variable{output/inst-diam.txt}}
\newcommand{\lifetime}{\variable{output/inst-lifetime.txt}}
\newcommand{\ucen}{\variable{output/inst-center_u.txt}}
\newcommand{\bcen}{\variable{output/inst-center_b.txt}}
\newcommand{\vcen}{\variable{output/inst-center_v.txt}}

\newcommand{\uwidth}{\variable{output/inst-width_u.txt}}
\newcommand{\bwidth}{\variable{output/inst-width_b.txt}}
\newcommand{\vwidth}{\variable{output/inst-width_v.txt}}

\newcommand{\uthrough}{\variable{output/inst-through_u.txt}}
\newcommand{\bthrough}{\variable{output/inst-through_b.txt}}
\newcommand{\vthrough}{\variable{output/inst-through_v.txt}}


% Title
\title{The Stratospheric Ozone Ultraviolet Photometer (SOUP) Mission}

% Author list
\author{Ted Johnson}
\affiliation{UNLV}
\email{ted.johnson@unlv.edu}

% Abstract with filler text
\begin{abstract}
    We present the Stratospheric Ozone Ultraviolet Photometer (SOUP) mission concept, a space telescope designed to detect the biosignature ozone in the upper atmospheres of rocky exoplanets. SOUP combines the sensitivity of transmission spectroscopy with the efficiency of a photometric survey by taking high-precision, full field-of-view images in three photometric bands.
\end{abstract}

\section{Introduction}
One of the most powerful methods for characterizing the composition of exoplanet atmospheres to date is transmission spectroscopy \citep[e.g.][]{garcia2022,may2023,moran2023,cadieux2024,fournier-tondreau2024}. Despite this technique's unique ability to probe the gasses in the upper atmospheres of a wide variety of exoplanets, its immense cost prevents its meaningful use on rocky, habitable-zone (HZ) exoplanets which would require hundreds of observations to reach the signal-to-noise ratio (SNR) required to detect a biosignature.

Conversely, thousands\footnote{\url{https://exoplanetarchive.ipac.caltech.edu/docs/counts_detail.html}} of exoplanet have been found via the transit method because it relies on photometry -- a dedicated mission (e.g. Kepler \citep{borucki2004} or TESS \citep{winn2024}) can observe thousands of targets at once. I propose a new mission to combine the best elements of each technique in order to efficiently explore a whole population of Earth-like planets in search of the biosignature ozone. I call this mission the Stratospheric Ozone Ultraviolet Photometer, or SOUP. Figure \ref{fig:soup-spec} demonstrates the basic idea behind detecting \oz{} through photometry: if we measure the transit depth simultaneously in multiple bands, we can infer the presence of absorbing gasses in the atmosphere. When we chose these bands carefully, we can become sensitive to potential biosignatures such as \oz{}.

\begin{figure}
    \includegraphics[width=0.5\textwidth]{figures/spectrum.pdf}
    \caption{Transmission spectra and band-integrated transit depths for two sets of potential SOUP targets. {\bf Top}: Earth-sized planet orbiting an M-dwarf at a distance of 10 parsecs from the solar system. The green ``Has \oz'' model has an ozone volume-mixing ratio (VMR) of $10^{-4}$. Both models include a background \ce{N2} atmosphere. {\bf Bottom}: Sub-Neptune orbiting a Sun-like star at a distance of 200 parsecs. Like the panel above, ``Has \oz'' means a VMR of $10^{-4}$, but the background gas in this case is \ce{H2}. In both panels, the planet has an isothermal 300 K atmosphere and an opaque surface at 1 bar. The noise in these models is calculated based on the number of transits observed over the \lifetime{}-year mission lifetime. The slope that appears in all spectra is due to scattering of the background gas and behaves predictably.}
    \label{fig:soup-spec}
    \script{spectrum.py}
\end{figure}

\subsection{Ozone as a biosignature}

\subsection{Abiotic Ozone}

\subsection{Transit Spectroscopy in the Era of JWST}

\subsection{Multiband Photometry}

\section{Mission Design}

\subsection{Science Goals}

\subsubsection{Primary Goal: Detection of Ozone}

\subsection{Secondary Goal: Atmosphere Detection}

\subsection{Secondary Goal: Planet Discovery}

\subsection{Secondary Goal: Stellar Characterization}

\subsection{Secondary Goal: Transient Event Science}

\subsection{Instrument}

The SOUP spacecraft will consist of a \diam{} aperture mirror that feeds light through a series of dichromatic beam splitters. The first splitter reflects UV light through the U filter onto a detector. The transmitted light passes through a second splitter, which transmits light short of 500 nm. The reflected light passes through the V filter and transmitted light through the B filter, both beams landing on an optical detector. These three photometric bands are described in Table \ref{tab:bands}
\begin{table}
    \centering
    \begin{tabular}{c|ccc}
        \hline
        Band & Center & Width & Throughput \\
        \hline
        U & \ucen & \uwidth & \uthrough \\
        B & \bcen & \bwidth & \bthrough \\
        V & \vcen & \vwidth & \vthrough \\ \hline \hline
        
    \end{tabular}
    \caption{The SOUP photometric bands. The throughput column accounts of the efficiency of the optics and detector quantum efficiency averaged over the bandpass.}
    \label{tab:bands}
\end{table}

\subsection{Orbit \& Logistics}

\subsection{Targets}

\subsection{Data Processing}

\section{Mission Yield}

\subsection{Ozone Detection}

\subsection{Secondary Goals}

\section{Budget}

\section{Timeline}


\bibliographystyle{aasjournal}
\bibliography{vspec,soup}

\end{document}
